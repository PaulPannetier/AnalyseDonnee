% Options for packages loaded elsewhere
\PassOptionsToPackage{unicode}{hyperref}
\PassOptionsToPackage{hyphens}{url}
%
\documentclass[
]{article}
\usepackage{amsmath,amssymb}
\usepackage{lmodern}
\usepackage{iftex}
\ifPDFTeX
  \usepackage[T1]{fontenc}
  \usepackage[utf8]{inputenc}
  \usepackage{textcomp} % provide euro and other symbols
\else % if luatex or xetex
  \usepackage{unicode-math}
  \defaultfontfeatures{Scale=MatchLowercase}
  \defaultfontfeatures[\rmfamily]{Ligatures=TeX,Scale=1}
\fi
% Use upquote if available, for straight quotes in verbatim environments
\IfFileExists{upquote.sty}{\usepackage{upquote}}{}
\IfFileExists{microtype.sty}{% use microtype if available
  \usepackage[]{microtype}
  \UseMicrotypeSet[protrusion]{basicmath} % disable protrusion for tt fonts
}{}
\makeatletter
\@ifundefined{KOMAClassName}{% if non-KOMA class
  \IfFileExists{parskip.sty}{%
    \usepackage{parskip}
  }{% else
    \setlength{\parindent}{0pt}
    \setlength{\parskip}{6pt plus 2pt minus 1pt}}
}{% if KOMA class
  \KOMAoptions{parskip=half}}
\makeatother
\usepackage{xcolor}
\usepackage[margin=1in]{geometry}
\usepackage{color}
\usepackage{fancyvrb}
\newcommand{\VerbBar}{|}
\newcommand{\VERB}{\Verb[commandchars=\\\{\}]}
\DefineVerbatimEnvironment{Highlighting}{Verbatim}{commandchars=\\\{\}}
% Add ',fontsize=\small' for more characters per line
\usepackage{framed}
\definecolor{shadecolor}{RGB}{248,248,248}
\newenvironment{Shaded}{\begin{snugshade}}{\end{snugshade}}
\newcommand{\AlertTok}[1]{\textcolor[rgb]{0.94,0.16,0.16}{#1}}
\newcommand{\AnnotationTok}[1]{\textcolor[rgb]{0.56,0.35,0.01}{\textbf{\textit{#1}}}}
\newcommand{\AttributeTok}[1]{\textcolor[rgb]{0.77,0.63,0.00}{#1}}
\newcommand{\BaseNTok}[1]{\textcolor[rgb]{0.00,0.00,0.81}{#1}}
\newcommand{\BuiltInTok}[1]{#1}
\newcommand{\CharTok}[1]{\textcolor[rgb]{0.31,0.60,0.02}{#1}}
\newcommand{\CommentTok}[1]{\textcolor[rgb]{0.56,0.35,0.01}{\textit{#1}}}
\newcommand{\CommentVarTok}[1]{\textcolor[rgb]{0.56,0.35,0.01}{\textbf{\textit{#1}}}}
\newcommand{\ConstantTok}[1]{\textcolor[rgb]{0.00,0.00,0.00}{#1}}
\newcommand{\ControlFlowTok}[1]{\textcolor[rgb]{0.13,0.29,0.53}{\textbf{#1}}}
\newcommand{\DataTypeTok}[1]{\textcolor[rgb]{0.13,0.29,0.53}{#1}}
\newcommand{\DecValTok}[1]{\textcolor[rgb]{0.00,0.00,0.81}{#1}}
\newcommand{\DocumentationTok}[1]{\textcolor[rgb]{0.56,0.35,0.01}{\textbf{\textit{#1}}}}
\newcommand{\ErrorTok}[1]{\textcolor[rgb]{0.64,0.00,0.00}{\textbf{#1}}}
\newcommand{\ExtensionTok}[1]{#1}
\newcommand{\FloatTok}[1]{\textcolor[rgb]{0.00,0.00,0.81}{#1}}
\newcommand{\FunctionTok}[1]{\textcolor[rgb]{0.00,0.00,0.00}{#1}}
\newcommand{\ImportTok}[1]{#1}
\newcommand{\InformationTok}[1]{\textcolor[rgb]{0.56,0.35,0.01}{\textbf{\textit{#1}}}}
\newcommand{\KeywordTok}[1]{\textcolor[rgb]{0.13,0.29,0.53}{\textbf{#1}}}
\newcommand{\NormalTok}[1]{#1}
\newcommand{\OperatorTok}[1]{\textcolor[rgb]{0.81,0.36,0.00}{\textbf{#1}}}
\newcommand{\OtherTok}[1]{\textcolor[rgb]{0.56,0.35,0.01}{#1}}
\newcommand{\PreprocessorTok}[1]{\textcolor[rgb]{0.56,0.35,0.01}{\textit{#1}}}
\newcommand{\RegionMarkerTok}[1]{#1}
\newcommand{\SpecialCharTok}[1]{\textcolor[rgb]{0.00,0.00,0.00}{#1}}
\newcommand{\SpecialStringTok}[1]{\textcolor[rgb]{0.31,0.60,0.02}{#1}}
\newcommand{\StringTok}[1]{\textcolor[rgb]{0.31,0.60,0.02}{#1}}
\newcommand{\VariableTok}[1]{\textcolor[rgb]{0.00,0.00,0.00}{#1}}
\newcommand{\VerbatimStringTok}[1]{\textcolor[rgb]{0.31,0.60,0.02}{#1}}
\newcommand{\WarningTok}[1]{\textcolor[rgb]{0.56,0.35,0.01}{\textbf{\textit{#1}}}}
\usepackage{graphicx}
\makeatletter
\def\maxwidth{\ifdim\Gin@nat@width>\linewidth\linewidth\else\Gin@nat@width\fi}
\def\maxheight{\ifdim\Gin@nat@height>\textheight\textheight\else\Gin@nat@height\fi}
\makeatother
% Scale images if necessary, so that they will not overflow the page
% margins by default, and it is still possible to overwrite the defaults
% using explicit options in \includegraphics[width, height, ...]{}
\setkeys{Gin}{width=\maxwidth,height=\maxheight,keepaspectratio}
% Set default figure placement to htbp
\makeatletter
\def\fps@figure{htbp}
\makeatother
\setlength{\emergencystretch}{3em} % prevent overfull lines
\providecommand{\tightlist}{%
  \setlength{\itemsep}{0pt}\setlength{\parskip}{0pt}}
\setcounter{secnumdepth}{-\maxdimen} % remove section numbering
\ifLuaTeX
  \usepackage{selnolig}  % disable illegal ligatures
\fi
\IfFileExists{bookmark.sty}{\usepackage{bookmark}}{\usepackage{hyperref}}
\IfFileExists{xurl.sty}{\usepackage{xurl}}{} % add URL line breaks if available
\urlstyle{same} % disable monospaced font for URLs
\hypersetup{
  pdftitle={main.Rmd},
  hidelinks,
  pdfcreator={LaTeX via pandoc}}

\title{main.Rmd}
\author{}
\date{\vspace{-2.5em}2023-05-14}

\begin{document}
\maketitle

Nous allons présenter ici notre projet d'analyse de données sur une base
de données rassemblant de nonbreuses caractéristiques sur 20052 plats de
cuisines du monde entier. Cette base publique est disponible sur Kaggle
avec ce lien
\url{https://www.kaggle.com/code/upsylend/pr-diction-sur-des-recettes-de-cuisine/input}.

La base possède notamment une variable rating (entre 0.0 et 5.0)
représentant si un plat est apprécié/goutu, le but ce cette analyse est
de prédire la valeur de la variable rating en fonction des autres
variables ainsi que faire de la classification des plats afin d'en
regrouper certains.

La base de données ayant beaucoup de variables, nous avons décidé de ne
garder que les principales, de plus énormément de variables n'ont que 2
modalitées (comme la variable disant si oui ou non il y a du bacon dans
le plat) qui vaut 1 que très rarement.

Nous ne garderons que les variables : - rating (appréciation du plat) -
calories (énergie apporté par le plat) - protein (quantité de protéines)
- fat (quantité de graisses) - sodium (quantité de sel) - alcoholic (0
ou 1, présence d'alcool) - bake (0 ou 1, plat rotie)

En supprimant les lignes avec au moins une valeur manquante sur ces
variable nous n'avons plus que 15864 plats disponibles pour notre
études.

\begin{Shaded}
\begin{Highlighting}[]
\FunctionTok{library}\NormalTok{(tidyr)}
\FunctionTok{rm}\NormalTok{(}\AttributeTok{list=}\FunctionTok{ls}\NormalTok{())}
\NormalTok{data }\OtherTok{=} \FunctionTok{read.csv}\NormalTok{(}\StringTok{"epi\_r.csv"}\NormalTok{, }\AttributeTok{sep =}\StringTok{","}\NormalTok{)}
\NormalTok{data }\OtherTok{=}\NormalTok{ data[}\FunctionTok{c}\NormalTok{(}\DecValTok{2}\NormalTok{,}\DecValTok{3}\NormalTok{,}\DecValTok{4}\NormalTok{,}\DecValTok{5}\NormalTok{,}\DecValTok{6}\NormalTok{,}\DecValTok{7}\NormalTok{,}\DecValTok{15}\NormalTok{,}\DecValTok{39}\NormalTok{)]}
\NormalTok{data }\OtherTok{=}\NormalTok{ data[data}\SpecialCharTok{$}\NormalTok{calories }\SpecialCharTok{!=} \StringTok{""}\NormalTok{,]}
\NormalTok{data }\OtherTok{=}\NormalTok{ data[data}\SpecialCharTok{$}\NormalTok{protein }\SpecialCharTok{!=} \StringTok{""}\NormalTok{,]}
\NormalTok{data }\OtherTok{=}\NormalTok{ data[data}\SpecialCharTok{$}\NormalTok{fat }\SpecialCharTok{!=} \StringTok{""}\NormalTok{,]}
\NormalTok{data }\OtherTok{=}\NormalTok{ data[data}\SpecialCharTok{$}\NormalTok{sodium }\SpecialCharTok{!=} \StringTok{""}\NormalTok{,]}
\NormalTok{data }\OtherTok{=} \FunctionTok{drop\_na}\NormalTok{(data)}
\CommentTok{\#data = data[c(1:5000),]}
\FunctionTok{head}\NormalTok{(data)}
\end{Highlighting}
\end{Shaded}

\begin{verbatim}
##   rating calories protein fat sodium X.cakeweek alcoholic bake
## 1  2.500      426      30   7    559          0         0    0
## 2  4.375      403      18  23   1439          0         0    1
## 3  3.750      165       6   7    165          0         0    0
## 5  3.125      547      20  32    452          0         0    1
## 6  4.375      948      19  79   1042          0         0    0
## 9  4.375      170       7  10   1272          0         0    0
\end{verbatim}

\begin{enumerate}
\def\labelenumi{\arabic{enumi})}
\tightlist
\item
  Analyse des données\textless{}\b>
\end{enumerate}

\begin{Shaded}
\begin{Highlighting}[]
\FunctionTok{boxplot}\NormalTok{(data)}
\end{Highlighting}
\end{Shaded}

\includegraphics{rapport_files/figure-latex/unnamed-chunk-2-1.pdf}

\end{document}
